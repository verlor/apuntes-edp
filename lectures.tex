\documentclass[10pt,spanish,a4paper]{book}
\usepackage[utf8]{inputenc}
%\usepackage[spanish]{babel}
\usepackage{amsmath}
\usepackage{amsfonts}
\usepackage{amssymb}
\author{Hugo,Noe,Carlos}
\title{Apuntes Ecuaciones Diferenciales Parciales}
\begin{document}

	\title{%
	Lecturas acerca de Ecuaciones Diferenciales Parciales }
	\author{Hugo Francisco Martinez Ortiz}
	\maketitle
	
	\chapter{Introducción}
		Una intro bien chidolira
		\section{Tipos de ecuaciones?}
		A lo mejor lleva algo, who knows?

	\chapter{Método de las características}
	Considerese la ecuación diferencial parcial
	\[ a(x,y)\frac{\partial u}{\partial x} + b(x,y)\frac{\partial u}{\partial y} + c(x,y)u
	   = d(x,y) \]
	que en adelante se expresará como:
	\[ a(x,y)u_{x} + b(x,y)u_{y} + c(x,y)u
	   = d(x,y) \]
	Se busca una transformación de coordenadas que pueda reducir esta ecuación diferencial parcial a una ecuación diferencial ordinaria. De manera que se propone la siguiente transformación:
	\[\xi = \phi(x,y) \]
	\[\eta = \psi(x,y) \]
	\[u(x,y) = w(\xi,\eta)\]
	donde $\frac{\partial(\xi,\eta)}{\partial(x,y)} \not= 0 \therefore$ la transformada es invertible. \\
	Calculando $u_{x} $ y $ u_{y}$ de acuerdo a la nueva transformación
	\[u_{x} = w_{\xi}\xi_{x} + w_{\eta}\eta_{x}\]
	\[u_{y} = w_{\xi}\xi_{y} + w_{\eta}\eta_{y}\]
	Sustituyendo en la ecuación diferencial parcial original
	\[a(w_{\xi}\xi_{x} + w_{\eta}\eta_{x}) + b(w_{\xi}\xi_{y} + w_{\eta}\eta_{y}) + cw = \tilde{d}\]
	y reordenando los términos
	\[w_{\xi}(a\xi_{x} + b\xi_{y}) + w_{\eta}(a\eta_{x} + b\eta_{y}) + cw = \title{d} \]
	El objetivo de este cambio de coordenadas es obtener una ecuación diferencial ordinaria, de manera se propone que
	\[a\eta_{x} + b\eta_{y} = 0 \Rightarrow a\eta_{x} = -b\eta_{y} \]
	es importante notar que para que la igual propuesta sea cierta, la función
	\[\eta(x,y) = c\]
	debe ser constante.
\end{document}